\documentclass{article}

\usepackage[french, english]{babel}
\usepackage[utf8]{inputenc}
\usepackage[T1]{fontenc}

%% Better maths
\usepackage{amsmath}
\usepackage{amssymb}

%% Code listings
\usepackage{listings}
\lstset{language=[Objective]Caml, frame=none, captionpos=b}

\usepackage{multicol}

\title{\textbf{Inhabitation Problem for lambda calculus with non-idempotent intersection types}}
\author{Ulysse Gérard}
\date\today

\begin{document}

\maketitle

\section{Introduction}


\section{Implementation}
\subsection{Types in OCaml}
The implementation of the inhabitation algorithm uses basic OCaml sum types with some polymorphism and recursivity.

\begin{lstlisting}[caption={Multisets}]
type 'a multiset =
  Empty
| Cons of 'a * 'a multiset
\end{lstlisting}

\begin{lstlisting}[caption={$\lambda$-calculus}]
type term = 
  Var of char
| Lambda of char * term 
| App of term * term
\end{lstlisting}

\begin{lstlisting}[caption={Intersection types and environments}]
type multisetType = 
  sType multiset
and sType = 
  Var of char
| Fleche of multisetType * sType

type environment = (char * multisetType) list
\end{lstlisting}

\begin{multicols}{3}[Titre sur une seule colonne.]
\begin{lstlisting}[caption={Approximate normal forms}]
type anf = 
  Omega
| N of n
\end{lstlisting}
\columnbreak
\begin{lstlisting}[caption={Approximate normal forms}]
and n = 
  Lambda of char * n
| L of l
\end{lstlisting}
\columnbreak
\begin{lstlisting}[caption={Approximate normal forms}]
and l =
  Var of char
| App of l * anf
\end{lstlisting}
\end{multicols}



\end{document}